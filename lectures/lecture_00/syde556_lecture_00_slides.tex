% !TeX spellcheck = en_GB
\documentclass[aspectratio=169]{beamer}
\input{../syde556_lecture_slides_preamble}

\date{September 4, 2024}
\title{SYDE 556/750 \\ Simulating Neurobiological Systems \\ Lecture 0: Administrative Remarks}

\begin{document}
	
\begin{frame}{}
	\MakeTitle
\end{frame}

\begin{frame}{Warning}
	\begin{columns}[T]
		\column{0.6\textwidth}
		\fboxrule=0.4pt\fboxsep=0pt\fbox{\includegraphics[height=0.8\textheight]{media/syde556-reviews.png}}\\
		\column{0.4\textwidth}
		\begin{itemize}
			\item The UWFlow reviews are accurate. 
			\item This can be a challenging course.
			\item Be prepared to spend a lot of time on the assignments.
			\item We'll be making use of pretty much everything in undergrad engineering,
			      and applying it to cognitive science and neuroscience.
		\end{itemize}
	\end{columns}
\end{frame}

\begin{frame}{Organization (I)}
	\begin{block}{Instructor}
		\vspace{2mm}
		\textbf{Chris Eliasmith}\\[2mm]
		\hspace{-2.5mm}\begin{tabular}{l l}
			Email & \url{celiasmith@uwaterloo.ca}\\
			Website & \url{compneuro.uwaterloo.ca}\\
		\end{tabular}
	\end{block}
 
	\vfill

	\begin{block}{Course website}
		\begin{itemize}
      \item Syllabus, project description, due dates: \url{http://compneuro.uwaterloo.ca/courses/syde-750.html}
			\item Assignments, slides, lecture notes: \url{https://github.com/celiasmith/syde556-f24}
		\end{itemize}
	\end{block}
\end{frame}

\begin{frame}{Organization (II)}
	\begin{block}{Course times and logistics - All meetings in E5-6008}
		\begin{itemize}
			\item \textbf{Monday:}\\
			12:30-2:20 Lecture
			\item \textbf{Wednesday:}\\
      2:30-3:20 Lecture
			\item \textbf{Wednesday:}\\
			3:30-4:20 In person discussion (SYDE 750, optional for 556) 
		\end{itemize}
	\end{block}

\end{frame}

\begin{frame}{Textbooks and Readings}
	\begin{columns}[T]
		\column{0.2\textwidth}
			\raggedleft
			\fboxrule=0.4pt\fboxsep=0pt\fbox{\includegraphics[height={\dimexpr 0.4\textheight - 0.8pt}]{media/neural_engineering_cover.png}}\\
		\column{0.3\textwidth}  
			\small
			\textbf{Main text:}\\
			Chris Eliasmith and\\Charles H. Anderson\\
			\emph{Neural Engineering: Computation, Representation, and Dynamics in Neurobiological Systems}, MIT Press, 2003.
		\column{0.2\textwidth}
			\raggedleft
			\includegraphics[height=0.4\textheight]{media/how_to_build_a_brain_cover.png}\\
		\column{0.3\textwidth}
			\small
			\textbf{Optional:}\\
			Chris Eliasmith\\
			\emph{How to Build a Brain}, Oxford University Press, 2013.
	\end{columns}
\end{frame}

\begin{frame}{Coursework (SYDE 556 \& SYDE 750)}
	\begin{columns}[t]
		\column{0.5\textwidth}
		\begin{block}{\hl{Five Assignments}}
		\begin{itemize}
				\item 20\%, 20\%, 15\%, 15\%, 30\%, respectively
				\item Roughly two weeks for each assignment
				\item Everyone must write their own code, generate their own graphs, and write their own answers.
			\end{itemize}
		\end{block}
		\column{0.5\textwidth}
		\begin{block}{\hl{750 Final Project} (SYDE 556 optional, replaces assn 5)}
			\begin{itemize}
				\item Build a model of some neural system.
				\item Replicable science: report everything needed to recreate your model and analysis
				\item 20\% of grade (assignments are rescaled to 80\%)
				\item Have your project proposal approved via email by Oct 23rd (see template)
			\end{itemize}
		\end{block}
	\end{columns}
\end{frame}

\begin{frame}{Coursework (SYDE 750 only)}
	\begin{block}{Class Participation in the Seminar  (SYDE 750 only; optional for SYDE 556)}
	\begin{itemize}
		\item General discussion about Neuroscience, cognitive science, AI, etc.
		\item Each student is asked to submit questions or interesting observations pertaining to this week's reading, lecture notes, or the material referenced in the lecture (this should be about 100 words).
		\item Questions must be submitted via email to the instructor (\url{celiasmith@uwaterloo.ca}) by midnight (23:59 EST) on the Tuesday before.
    \item This is to ensure a lively discussion in the seminar --- there are no explicit marks for this part of the course, but participation will be incorporated into the final project mark.
  \end{itemize}
	\end{block}
\end{frame}

\begin{frame}{Schedule}
    \begin{itemize}
      \item See here: \url{http://compneuro.uwaterloo.ca/courses/syde-750/syde-556-course-outline.html}
      \item Official syllabus: \url{https://outline.uwaterloo.ca/view/nwh9u7}
    \end{itemize}
\end{frame}

\begin{frame}{To get started}
  \begin{itemize}
    \item Get the textbook (\enquote{Neural Engineering}, Chris Eliasmith and Charles Anderson, 2003)
    \item Be able to run \texttt{jupyter lab} or \texttt{jupyter notebook} with a Python 3 kernel. Install \texttt{numpy}, \texttt{scipy}, and \texttt{matplotlib}. \href{https://www.anaconda.com/distribution/}{Anaconda} is a Python distribution that ships with these packets preinstalled, so (depending on your platform) this might be the easiest to use.
    \item Start thinking about a project\textellipsis already.
  \end{itemize}
\end{frame}

\end{document}