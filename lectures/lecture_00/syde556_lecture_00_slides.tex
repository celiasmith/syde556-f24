% !TeX spellcheck = en_GB
\documentclass[aspectratio=169]{beamer}
\input{../syde556_lecture_slides_preamble}

\date{September 8, 2021}
\title{SYDE 556/750 \\ Simulating Neurobiological Systems \\ Lecture 0: Administrative Remarks}

\begin{document}
	
\begin{frame}{}
	\MakeTitle
\end{frame}

\begin{frame}{Warning}
	\begin{columns}[T]
		\column{0.6\textwidth}
		\fboxrule=0.4pt\fboxsep=0pt\fbox{\includegraphics[height=0.8\textheight]{media/syde556-reviews.png}}\\
		\column{0.4\textwidth}
		\begin{itemize}
			\item The UWFlow reviews are accurate. 
			\item This is a tough course.
			\item Be prepared to spend a lot of time on the assignments.
			\item We'll be making use of pretty much everything in the SyDe undergrad
			      program, and applying it to cognitive science and neuroscience.
			\item Unique course on an approach developed here at Waterloo by a SyDe graduate.
		\end{itemize}
	\end{columns}
\end{frame}

\begin{frame}{About Me}
	\begin{columns}[T]
		\column{0.3\textwidth}
		\fboxrule=0.4pt\fboxsep=0pt\fbox{\includegraphics[height=0.8\textheight]{media/TerryStewart2.png}}\\
		\column{0.7\textwidth}
		\begin{itemize}
			\item Terry Stewart
			\item Research Officer at the National Research Council Canada (NRC)
			\begin{itemize}
				\item Investigate algorithms underlying biological cognition
				\item Build computational models of them
				\item Determine if they may be useful to industry
			\end{itemize}
			\item Undergrad: Systems Design Engineering at Waterloo
			\item Masters: M.Phil in Comp.Sci and AI at Sussex University (UK)
			\item PhD: Cognitive Science at Carleton University (Ottawa)
			\item Post-doc: at Waterloo, working with Chris Eliasmith on the research discussed in this course.
		\end{itemize}
	\end{columns}
\end{frame}



\begin{frame}{Organization (I)}
	\begin{block}{Instructor}
		\vspace{2mm}
		\textbf{Terry Stewart}\\[2mm]
		\hspace{-2.5mm}\begin{tabular}{l l}
			Email & \url{terry.stewart@gmail.com}\\
			Website & \url{https://terrystewart.ca}\\
		\end{tabular}
	\end{block}
 
	\vfill

	\begin{block}{Course website}
		\begin{itemize}
			\item LEARN
			\item \url{https://github.com/tcstewar/syde556-f21}
			\item \url{syde556-f21.slack.com}
		\end{itemize}
	\end{block}
\end{frame}

\begin{frame}{Organization (II)}
	\begin{block}{Course times and logistics}
		\begin{itemize}
			\item \textbf{Saturday:}\\
			Pre-recorded lectures posted
			\item \textbf{Monday:}\\
			8:30-9:50 online lecture and discussion (LEARN)
			\item \textbf{Tuesday:}\\9:00-9:50 online discussion (LEARN) (SYDE 750, optional for 556)
			\item \textbf{Wednesday:}\\
			8:30-9:50 online lecture and discussion (LEARN) 
			\item \textbf{Any time:}\\
Email \url{terry.stewart@gmail.com}\\ Slack \url{syde556750-f21.slack.com} 
		\end{itemize}
	\end{block}

\end{frame}

\begin{frame}{Textbooks and Readings}
	\begin{columns}[T]
		\column{0.2\textwidth}
			\raggedleft
			\fboxrule=0.4pt\fboxsep=0pt\fbox{\includegraphics[height={\dimexpr 0.4\textheight - 0.8pt}]{media/neural_engineering_cover.png}}\\
		\column{0.3\textwidth}  
			\small
			\textbf{Main text:}\\
			Chris Eliasmith and\\Charles H. Anderson\\
			\emph{Neural Engineering: Computation, Representation, and Dynamics in Neurobiological Systems}, MIT Press, 2003.
		\column{0.2\textwidth}
			\raggedleft
			\includegraphics[height=0.4\textheight]{media/how_to_build_a_brain_cover.png}\\
		\column{0.3\textwidth}
			\small
			\textbf{Optional:}\\
			Chris Eliasmith\\
			\emph{How to Build a Brain}, Oxford University Press, 2013.
	\end{columns}
\end{frame}

\begin{frame}{Coursework (SYDE 556 \& SYDE 750)}
	\begin{columns}[t]
		\column{0.5\textwidth}
		\begin{block}{\hl{Five Assignments}}
		\begin{itemize}
				\item 20\%, 20\%, 15\%, 15\%, 30\%, respectively
				\item Roughly two weeks for each assignment
				\item Everyone must write their own code, generate their own graphs, and write their own answers.
			\end{itemize}
		\end{block}
		\column{0.5\textwidth}
		\begin{block}{\hl{Final Project} (SYDE 750 only)}
			\begin{itemize}
				\item Build a model of some neural system.
				\item Replicable science: report everything needed to recreate your model and analysis
				\item 20\% of grade (assignments are rescaled to 80\%)
				\item Have your project proposal approved via email by Oct 27
			\end{itemize}
		\end{block}
	\end{columns}
\end{frame}

\begin{frame}{Coursework (SYDE 750 only)}
	\begin{block}{Class Participation in the Seminar  (SYDE 750 only; optional for SYDE 556)}
	\begin{itemize}
		\item General discussion about Neuroscience, cognitive science, AI, etc.
		\item Special interest: replicable science and computational modelling
		\item SYDE 750 students must attend the seminar (Tuesday, 9:00-9:50).
		\item No marks for this part of the course.
	\end{itemize}
	\end{block}
\end{frame}

\begin{frame}{Schedule (I)}
\small
\begin{tabular}{p{2cm} p{2cm} p{5cm} p{3cm}}
	\toprule
	\textbf{Date} &	\textbf{Reading} &	\textbf{Topic} & \textbf{Assignments} \\
	\tiny WEEK 1 & & & \\
	Sept 8 &
	Chapter 1 &
	Introduction &
	\\[0.05cm]
	\tiny WEEK 2 & & & \\
	Sept 13 &
	Chapter 2 &
	Neurons &
	\\
	Sept 15 &
	Chapter 2 &
	Population Representation (I) &
	\#1 posted
	\\[0.05cm]
	\tiny WEEK 3 & & & \\
	Sept 20 &
	Chapter 2 &
	Population Representation (II) &
	\\
	Sept 22 &
	Chapter 4 &
	Temporal Representation &
	\\[0.05cm]
	\tiny WEEK 4 & & & \\
	Sept 27 &
	 &
	Guest Lecture &
	\\
	Sept 29 & & Guest Lecture &
	\\[0.05cm]
	\tiny WEEK 5 & & & \\
	
	Oct 4 &
	Chapters 5, 6 &
	Feedforward Transformations (I) &
	\#1 due*, \#2 posted\\
	Oct 6 &
	Chapters 5, 6 &
	Feedforward Transformations (II) &
	\\[0.05cm]
	\tiny WEEK 6 & \multicolumn{3}{c}{\emph{--- Reading week, no lectures ---}} \\[0.05cm]
	
	\bottomrule
\end{tabular}
\end{frame}


\begin{frame}{Schedule (II)}
	\small
	\begin{tabular}{p{2cm} p{2cm} p{5cm} p{3cm}}
		\toprule
		\textbf{Date} &	\textbf{Reading} &	\textbf{Topic} & \textbf{Assignments} \\
		
		\tiny WEEK 7 & & & \\
		Oct 18 &
		Chapter 8 &
		Dynamics (I) &
		\\
		Oct 20 &
		Chapter 8 &
		Dynamics (II) &
		\\
		
		\tiny WEEK 8 & & & \\
		Oct 25 &
		Chapter 7 &
		Analysis of Representation &
		\#2 due*, \#3 posted\\
		Oct 27 &
		\emph{provided} &
		Temporal Basis Functions &
		\\
		&
		&
		&
		SYDE 750 Project proposal due\\[0.05cm]
		
		\tiny WEEK 9 & & & \\
		Nov 1 &
		\emph{provided} &
		Symbols (I) &
		\\
		Nov 3 &
		\emph{provided} &
		Symbols (II) &
		\\[0.05cm]
		
		\tiny WEEK 10 & & & \\
		Nov 8 &
		Chapter 8 &
		Memory &
		\#3 due*, \#4 posted\\
		Nov 10 &
		\emph{provided} &
		Action Selection &
		
		\\
		\bottomrule
	\end{tabular}
\end{frame}

\begin{frame}{Schedule (III)}
	\small
	\begin{tabular}{p{2cm} p{2cm} p{5cm} p{3cm}}
		\toprule
		\textbf{Date} &	\textbf{Reading} &	\textbf{Topic} & \textbf{Assignments} \\
		\tiny WEEK 11 & & & \\
		Nov 15 &
		Chaper 9 &
		Learning (I) &
		\\
		Nov 17 &
		Chaper 9 &
		Learning (II) &
		\\[0.05cm]
		\tiny WEEK 12 & & & \\
		Nov 22 &
		\emph{provided} &
		Spatial Semantic Pointers &
		\#4 due*\\
		Nov 24 &
		\emph{provided} &
		Biological Details &
		\\[0.05cm]
		
		\tiny WEEK 13 & & & \\
		Nov 29 &
		\emph{provided} &
		Other modelling frameworks &
		\\
		Dec 1 &
		&
		Conclusion &
		\\[0.05cm]
		
		\tiny WEEK 14 & & & \\
		Dec 6 &
		&
		Discussion &
		\\[0.05cm]
		
		\tiny WEEK 16 & & & \\
		Dec 23 &
		&
		&
		\#5 due; SYDE 750 projects due* \\
		\bottomrule
	\end{tabular}\\[0.2cm]
	\footnotesize
	* The project and all assignments are due at midnight ($\approx$ 11:59p Eastern) of that day.
\end{frame}

\begin{frame}{Homework}
	\begin{itemize}
		\setlength{\itemsep}{0.5cm}
		\item \textbf{Get the \hl{textbook}, read the first chapter}\\
		(\enquote{Neural Engineering}, Chris Eliasmith and Charles Anderson, 2003)
		\item \textbf{Be able to run \hl{\texttt{jupyter lab}} or (\texttt{jupyter notebook}) with \hl{Python 3}}\\
		Install \texttt{numpy}, \texttt{scipy}, and \texttt{matplotlib}. You may want to use \href{https://www.anaconda.com/distribution/}{Anaconda}, which ships with these packets preinstalled.
	\end{itemize}
\end{frame}

\end{document}