% !TeX spellcheck = en_GB
\documentclass[10pt,letterpaper,oneside]{article}
\usepackage{fontspec}
\usepackage{arev}
\usepackage[utf8]{inputenc}
\usepackage[T1]{fontenc}
\usepackage{amsmath}
\usepackage{amsfonts}
\usepackage{amssymb}
\usepackage{graphicx}
\usepackage{csquotes}
\usepackage{booktabs}
\usepackage{multicol}
\usepackage{enumerate}
\usepackage{microtype}
\usepackage[labelfont=bf,font={small}]{caption}
\usepackage{hyperref}
\usepackage{booktabs}
\usepackage{subcaption}
\usepackage{fancyhdr}
\usepackage[svgnames]{xcolor}
\usepackage{mdframed}
\usepackage{multicol}
\usepackage[para]{footmisc}
\usepackage{siunitx}
\usepackage{cleveref}
\usepackage{listings}
\usepackage{cprotect}


\lstset{ % General setup for the package
	language=Python,
	basicstyle=\small\ttfamily,
	tabsize=4,
	columns=fixed,
	showstringspaces=false,
	showtabs=false,
	keepspaces,
	commentstyle=\color{SeaGreen},
	keywordstyle=\bf\ttfamily\color{DarkBlue},
	stringstyle=\ttfamily\color{Crimson}
}

\newfontfamily\symbolfont{Symbola}
\usepackage[left=1in,right=1in,top=1in,bottom=1in,marginparwidth=0.3in]{geometry}

\usepackage[sorting=none]{biblatex}
\addbibresource{../bibliography.bib}

\ifx\NoteAuthor\undefined
  \def\NoteAuthor{Andreas Stöckel and Chris Eliasmith}
\fi

\ifx\BasedOn\undefined
  \def\BasedOn{Based on lecture notes by\\Chris Eliasmith and Terrence~C.~Stewart}
\fi

\author{\NoteAuthor\\[0.5cm]\BasedOn}
\newcommand{\baseCodeURL}{https://github.com/celiasmith/syde556-f24/blob/master/lectures}

\fancyhf{}
\fancyhead[L]{SYDE 556/750 Lecture Notes}
\fancyhead[R]{\NoteAuthor}
\fancyfoot[C]{\thepage}
\pagestyle{fancy}

\setlength{\parindent}{0em}
\setlength{\parskip}{0.5em}
\renewcommand{\baselinestretch}{1.25}

\renewcommand{\vec}[1]{{\mathbf{#1}}}
\newcommand{\mat}[1]{{\mathbf{#1}}}
\newcommand{\T}{\ensuremath{\mathrm{T}}}
\renewcommand{\epsilon}{\varepsilon}
\renewcommand{\phi}{\varphi}

\makeatletter
\newcommand{\superimpose}[2]{%
	{\ooalign{{#1}\hidewidth\cr{#2}\hidewidth\cr}}}
\makeatother
\newcommand{\SolidCircle}[2]{\superimpose{\color{#1}\symbolfont ⬤}{\textbf{\color{white}#2}}\hspace{1em}}
\newcommand{\OPlus}{\SolidCircle{DarkGreen}{\kern0.75pt+}}
\newcommand{\OMeh}{\SolidCircle{DarkOrange}{~}}
\newcommand{\OMinus}{\SolidCircle{DarkRed}{\kern2.25pt--}}

\newcommand{\YouTube}[2][Video]{\href{https://youtu.be/#2}{{\symbolfont 📺}~{#1}}%
%\footnote{\url{https://youtu.be/#2}}%
}

\newcommand{\CodeLink}[2][Code]{\href{\baseCodeURL/#2}{{\symbolfont ⌨}~\emph{#1}}}

\newcommand{\MakeTitle}[1]{
\maketitle
\begin{center}
	\includegraphics[width=0.5\textwidth]{../assets/uwlogo.pdf}\\[1cm]
	{#1}\
\end{center}

\vfill

\thispagestyle{empty}
\setcounter{page}{0}
\newpage

\pagenumbering{roman}
\setcounter{tocdepth}{2}
\tableofcontents
\newpage

\setcounter{page}{0}
\pagenumbering{arabic}}

\reversemarginpar


\newcommand{\ColorBox}[3]{%
	\marginpar{%
		\huge\raisebox{-3ex}{\symbolfont{#1}}%
	}%
	\begin{mdframed}[hidealllines=true,backgroundcolor=#2,innertopmargin=0.25cm,innerbottommargin=0.25cm]%
		{#3}
	\end{mdframed}}

\newcommand{\Note}[1]{\ColorBox{📌}{WhiteSmoke}{\textbf{Note:} #1}}
\newcommand{\Example}[1]{\ColorBox{💡}{WhiteSmoke}{\textbf{Example:} #1}}
\newcommand{\Aside}[1]{\ColorBox{🌟}{WhiteSmoke}{\emph{Aside:} #1}}
\newcommand{\Python}[1]{\ColorBox{🐍}{WhiteSmoke}{#1}}
\newcommand{\Notation}[1]{\ColorBox{\huge$\Sigma$}{WhiteSmoke}{\textbf{Notaton:} #1}}

\newcommand{\ConstructionSite}{\hrulefill {\symbolfont 🚧} UNDER CONSTRUCTION {\symbolfont 🚧} \hrulefill}

\newenvironment{ImportantEqn}[1]{\mdframed\raggedleft\emph{({#1})}\align}{\endalign\endmdframed}

\date{Sept 4, 2024}
\title{SYDE 556/750 \\ Simulating Neurobiological Systems \\ Lecture 0: Administrative Remarks}

\begin{document}

\MakeTitle{\textbf{Course website:}\\\url{http://compneuro.uwaterloo.ca/courses/syde-750.html}}

\section{Organization}

\begin{itemize}
	\item \textbf{Course website}\\
		  Links to all course material, including slides and these lecture notes and slides can be found at the following URLs:
		  \begin{itemize}
		  	\item \url{http://compneuro.uwaterloo.ca/courses/syde-750.html}
		  	\item \url{https://github.com/celiasmith/syde556-f24}
		  \end{itemize}
		  \emph{Note:} Any material on GitHub should be considered \enquote{preliminary} until officially linked at from the course website. Until then, the material is still subject to change.
	\item \textbf{Instructor}\\
		  Chris Eliasmith\\
		  Office: E7-6324\\
		  Email: \url{celiasmith@uwaterloo.ca}

	\item \textbf{Readings}
		\begin{itemize}
			\item Main resource: \enquote{Neural Engineering}, Chris Eliasmith and Charles Anderson, 2003 \cite{eliasmith2003neural}
			\item Optional: \enquote{How to Build a Brain}, Chris Eliasmith, 2012 \cite{eliasmith2013how}
		\end{itemize}
\end{itemize}

\newpage

\section{Coursework}

\begin{itemize}
	\item \textbf{Five assignments} 
	\begin{itemize}
		\item The assignments are worth 20\%, 20\%, 15\%, 15\%, 30\% of the final mark, respectively. (For SYDE 750 students, the final project is worth 20\% of the final mark and these are scaled to 80\%.)
		\item You have about two weeks for each assignment.
		\item You are free to discuss the assignments with other students, but do not take any (written) notes during such discussions. Everyone must write their own code, generate their own graphs, and write their own answers.
		\item These assignments (particularly the first two) are a lot of work, so start early.
	\end{itemize}
	\item \textbf{Final project} (worth 30\% of the final mark for 556 and 20\% for 750)
	\begin{itemize}
		\item Optional for 556 students.
    \item Build a model of some neural system.
		\item This must be more of a research project with more novelty.
		\item Potential ideas are collected \href{http://compneuro.uwaterloo.ca/courses/syde-750/syde-556-possible-projects.html}{here}.
		\item In any case, your project idea needs to be approved via email before the date provided in the syllabus.
		\item See \href{http://compneuro.uwaterloo.ca/courses/syde-750/syde-556-possible-projects.html}{the project page} for more information.
	\end{itemize}
	\item \textbf{Class Participation in the Discussion}  (SYDE 750 only)
	\begin{itemize}
		\item SYDE 750 students must attend the weekly disucssion.
		\item Each student is asked to submit (at least) three questions or interesting observations pertaining this week's reading, lecture notes, or the material referenced in the lecture (this should be about 100 words).
		\item Questions must be submitted via email to the instructor by midnight (23:59 EST) on the day before the seminar.
		\item This is to ensure a lively discussion in the seminar.
	\end{itemize}
\end{itemize}

\newpage

\section{Things you should do to get started}

\begin{itemize}
	\item Get the textbook (\enquote{Neural Engineering}, Chris Eliasmith and Charles Anderson, 2003)
	\item Be able to run \texttt{jupyter lab} or \texttt{jupyter notebook} with a Python 3 kernel. Install \texttt{numpy}, \texttt{scipy}, and \texttt{matplotlib}. \href{https://www.anaconda.com/distribution/}{Anaconda} is a Python distribution that ships with these packets preinstalled, so (depending on your platform) this might be the easiest to use.
	\item Start thinking about a project\textellipsis already.
\end{itemize}

\printbibliography

\end{document}
