% !TeX spellcheck = en_GB
\ifcsname SlidesDistr\endcsname%
\documentclass[handout,aspectratio=169]{beamer}
\else%
\documentclass[aspectratio=169]{beamer}
\fi%
\input{../syde556_lecture_slides_preamble}

\date{October 9 \& 11, 2024}
\title{SYDE 556/750 \\ Simulating Neurobiological Systems \\ Lecture 6: Recurrent Dynamics}

\begin{document}
	
	\begin{frame}{}
		\vspace{0.5cm}
		\begin{columns}[c]
			\column{0.6\textwidth}
			\MakeTitle
			\column{0.4\textwidth}
			\includegraphics[width=\textwidth]{media/canada_150_mosaic_engine_small.jpg}
		\end{columns}
	\end{frame}

  \begin{frame}{NEF Principle 3: Dynamics}
		\centering
		\begin{mdframed}
			\textbf{NEF Principle 3 -- Dynamics}\\
			Neural dynamics are characterized by considering neural representations as control theoretic state variables. We can use control theory (and dynamical systems theory) to analyse and construct these systems.
		\end{mdframed}
	\end{frame}

	\begin{frame}{Feed Forward vs. Recurrent Connections}
		\centering
		\includegraphics{media/feed_forward_recurrent.pdf}
	\end{frame}

	\begin{frame}{Recurrence Experiments (I)}
		\centering
		\includegraphics[width=\textwidth]{media/fxp1_example.pdf}
	\end{frame}

	\begin{frame}{Recurrence Experiments (II)}
		\centering
		\includegraphics[width=\textwidth]{media/fmx_example.pdf}
	\end{frame}

	\begin{frame}{Recurrence Experiments (III)}
		\centering
		\includegraphics[width=\textwidth]{media/fxs_example.pdf}
	\end{frame}


	\begin{frame}{Making Sense of Dynamics}
		\centering
		\includegraphics[width=\textwidth]{media/synaptic_filter.pdf} 		
	\end{frame}

	%\begin{frame}{Phase Portraits}
%		\centering
%		\includegraphics[width=\textwidth]{media/phase_portraits.pdf}
%	\end{frame}

	\begin{frame}{Behaviour of a Recurrent Connection}
	Feed-forward
			\begin{align*}
				x(t) = g(u(t)) * h(t)				
			\end{align*}		
	Recurrent
			\begin{align*}
	x(t) &= (g(u(t)) + f(x(t))) * h(t)	\\
	X(s) &= (G(s) + F(s))H(s) \\
	X(s) &= (G(s) + F(s)){1 \over {1+s\tau}} \\
	X(s) + s\tau X(s) &= G(s) + F(s) \\
	s\tau X(s) &= G(s) + F(s) - X(s) \\
	sX(s) &= {{F(s) - X(s)} \over \tau} + {G(s) \over \tau} \\
	{dx \over dt} &= {{f(x(t))-x(t)} \over \tau} + {{g(u(t))} \over \tau}
			\end{align*}		
	
	\end{frame}


	\begin{frame}{Revisiting Recurrence Experiments}
		\centering
		\includegraphics[width=\textwidth]{media/phase_portraits.pdf} 		
	\end{frame}

	\begin{frame}{Behaviour of a Recurrent Connection}
	\begin{align*}
		{dx \over dt} &= {{f(x(t))-x(t)} \over \tau} + {{g(u(t))} \over \tau}
	\end{align*}		
	If we want this
	\begin{align*}
	{dx \over dt} &= a(x) + b(u)
	\end{align*}
	Then we set
	\begin{align*}
	f(x) &= \tau a(x) + x \\
	g(u) &= \tau b(u)
	\end{align*}
	And we get
	\begin{align*}
	{dx \over dt} &= {{(\tau a(x) + x) - x} \over \tau} + {{(\tau b(u))} \over \tau} \\
	{dx \over dt} &= {{a(x)}} + {{b(u)}}
	\end{align*}		
	
	
			
	
	
\end{frame}


	\begin{frame}{Implementing Dynamics using a Neural Ensemble}
		\centering
		\includegraphics[width=\textwidth]{media/lti_integrator_vs_neural.pdf} 
	\end{frame}

	\begin{frame}{Implementing Dynamical Systems as a Neural Ensemble}
		\begin{columns}[t]
			\column{0.5\textwidth}
			\centering
			\hl{LTI System}
			\begin{align*}
				\phi(\vec u, \vec x) &= \mat A \vec x + \mat B \vec u \\
				\phi'(\vec u, \vec x) &= \mat A' \vec x + \mat B' \vec u \\
				\mat A' &= \tau \mat A + \mat I \\ \mat B' &= \tau \mat B \,.
			\end{align*}
			\column{0.5\textwidth}
			\centering
			\hl{Additive Time-Invariant System}
			\begin{align*}
				\phi(\vec u, \vec x) &= f(\vec x) + g(\vec u) \\
				\phi'(\vec u, \vec x) &= f'(\vec x) + g'(\vec u) \\
				f'(\vec x) &= \tau f(\vec x) + \vec x \\
				g'(\vec u) &= \tau g(\vec u)
			\end{align*}
		\end{columns}
		\vspace{1cm}
		\centering \hl{\enquote{General} Recipe}\\[0.25cm]
		Scale the original dynamics by $\tau$, add feedback $\vec x$
	\end{frame}

	\begin{frame}{NEF Principle 3: Dynamics}
	\begin{columns}[b]
		\column{0.5\textwidth}
		\centering
		\hl{Time-Invariant Dynamical System}\\
		\begin{align*}
			\frac{\mathrm{d}\vec x(t)}{\mathrm{d}t} &= f(\vec x(t), \vec u(t))
		\end{align*}
		\column{0.5\textwidth}
		\centering
		\hl{Linear Time-Invariant (LTI)}
		\hl{Dynamical System}
		\begin{align*}
			\frac{\mathrm{d}\vec x(t)}{\mathrm{d}t} &= \mat A \vec x + \mat B \vec u
		\end{align*}
	\end{columns}
	\vspace{0.75cm}
	\begin{mdframed}
		\textbf{NEF Principle 3 -- Dynamics}\\
		Neural dynamics are characterized by considering neural representations as control theoretic state variables. We can use control theory (and dynamical systems theory) to analyse and construct these systems.
	\end{mdframed}
\end{frame}

\begin{frame}{Low-pass Filter Example}
Desired behaviour
\begin{align*}
	{dx \over dt} = {{u - x} \over \tau_{desired}}	
\end{align*}
Feed-forward input
\begin{align*}
	g(u) &= \tau_{synapse} ({u \over \tau_{desired}}) \\
	g(u) &= {\tau_{synapse} \over \tau_{desired}} u	
\end{align*}
Recurrent
\begin{align*}
	f(x) &= \tau_{synapse} ({{-x} \over \tau_{desired}}) + x \\
	f(x) &= (1-{\tau_{synapse} \over \tau_{desired}})x			
\end{align*}	
\end{frame}	

	\begin{frame}{Integrator Example (I)}
		\centering
		\includegraphics[width=\textwidth]{media/example_integrator.pdf}
	\end{frame}

	\begin{frame}{Integrator Example (II)}
		\centering
		\includegraphics[width=\textwidth]{media/example_integrator_phases.pdf}
	\end{frame}

	\begin{frame}{Oscillator Example (I)}
		\centering
		\includegraphics[width=\textwidth]{media/example_oscillator.pdf}
	\end{frame}

	\begin{frame}{Oscillator Example (II)}
		\centering
		\includegraphics[width=\textwidth]{media/example_oscillator_phases.pdf}
	\end{frame}

	\begin{frame}{Lorentz Attractor}
		\centering
		\includegraphics[width=\textwidth]{media/example_lorentz.pdf}
		\begin{align*}
			\frac{\mathrm{d}\vec x(t)}{\mathrm{d}t} &= \begin{pmatrix}
			10 x_2(t)-10x_1(t) \\
			-x_1(t) x_3(t)-x_2(t) \\
			x_1(t) x_2(t) - \frac{8}{3}(x_3(t)+28)-28
			\end{pmatrix}
		\end{align*}
	\end{frame}

	\begin{frame}{Heart Shape}
		\centering
		\includegraphics[width=\textwidth]{media/example_heart.pdf}
	\end{frame}

	\begin{frame}{Horizontal Eye Control}
		\centering
		\includegraphics[width=\textwidth]{media/example_eye_control.pdf}
	\end{frame}

	\backupbegin

	\begin{frame}[noframenumbering]{Image sources}
		\small
		\textbf{Title slide}\\\enquote{The Canada 150 Mosaic Mural}\\Author: Mosaic Canada Murals.\\From \href{https://commons.wikimedia.org/wiki/File:Canada_150_Mosaic_Engine.jpg}{Wikimedia}.
	\end{frame}


	\backupend
	
\end{document}
